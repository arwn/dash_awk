%******************************************************************************%
%                                                                              %
%                  libft.en.tex for libft                                      %
%                  Created on : Tue Apr 12 13:54:27 2016                       %
%                  Made by : Geoffrey "Geff" LEGER <vikingz@staff.42.fr>       %
%                                                                              %
%******************************************************************************%

\documentclass{42-en}


%******************************************************************************%
%                                                                              %
%                                   Prologue                                   %
%                                                                              %
%******************************************************************************%


\begin{document}


                              \title{Libft}
          \subtitle{Your first own library}
                 \member{Pedago}{pedago@42.fr}


\summary {
  The aim of this project is to code a \texttt{C} library regrouping
  usual functions that you'll be allowed to use in all your other
  projects.
}
    \maketitle

    \tableofcontents

\newpage
%******************************************************************************%
%                                                                              %
%                                  Foreword                                    %
%                                                                              %
%******************************************************************************%
\chapter{Foreword}

    This first project marks the beginning of your training
    to become de software engineer.\\

    To accompany you during this project, here is a list of
    oustanding music groups. It's highly probable that you won't
    like any of those. This will mean that you have poor music taste.
    I'm sure that you have some other qualities such as being able to
    hold your breath for more than 3 minutes or maybe you know by
    heart the names of the 206 United Nations' signatory states.
    The groups aren't listed in any particular order and
    the list does not need to be exhaustive. Click on the links
    to find out more.\\

    \begin{itemize}\itemsep7pt

        \item \href{https://www.youtube.com/watch?v=yqoLZpCexCk}{Between The Buried And Me}
        \item \href{https://www.youtube.com/watch?v=y3J-I7EWieg}{Between The Buried And Me, c'est bon, mangez-en}
        \item \href{https://www.youtube.com/watch?v=O-hnSlicxV4}{Tesseract}
        \item \href{http://www.youtube.com/watch?v=YecYHXhBE70}{Chimp Spanner}
        \item \href{http://www.youtube.com/watch?v=S4g7mPUskW8}{Emancipator}
        \item \href{http://www.youtube.com/watch?v=xcF6r4QVSXM}{Cynic}
        \item \href{http://www.youtube.com/watch?v=x0f_N6nTpmA}{Kalisia}
        \item \href{https://www.youtube.com/watch?v=k0NCIF7yWNs}{O.S.I}
        \item \href{http://www.youtube.com/watch?v=d6Y799Z7z-o}{Dream Theater}
        \item \href{http://www.youtube.com/watch?v=WO-lbBZbX24}{Pain Of Salvation}
        \item \href{http://www.youtube.com/watch?v=ig3L27s0BZg}{Crucified Barbara}\\

    \end{itemize}



\newpage
%******************************************************************************%
%                                                                              %
%                                 Introduction                                 %
%                                                                              %
%******************************************************************************%
\chapter{Introduction}

    The \texttt{libft} project builds on the concepts you learned
    during Day-06 of the bootcamp ie code a library of useful
    functions that you will be allowed to reuse in most of your
    \texttt{C} projects this year. This will save you a lot of
    precious time. The following assignments will have you write lines
    of code you already wrote during the bootcamp. See the
    \texttt{libft} project as a Bootcamp reminder and use it wisely to
    assess your level and progress.


            \begin{figure}[H]
                \begin{center}
                    \includegraphics[width=12cm]{biblio.jpg}
                    \caption{A possible representation of your Libft (artist's view)}
                \end{center}
            \end{figure}



%******************************************************************************%
%                                                                              %
%                                  Objectives                                  %
%                                                                              %
%******************************************************************************%
\chapter{Objectives}

    \texttt{C} programming can be very tedious when one doesn’t have
    access to those highly useful standard functions. This project
    makes you to take the time to re-write those functions, understand
    them, and learn to use them. This library will help you for all
    your future \texttt{C} projects.

    Through this project, we also give you the opportunity to
    expand the list of functions with your own. Take the
    time to expand your \texttt{libft} throughout the year.



%******************************************************************************%
%                                                                              %
%                             General Instructions                             %
%                                                                              %
%******************************************************************************%
\chapter{General Instructions}

    \begin{itemize}

        \item You must create the following functions in the order you
          believe makes most sense. We encourage you to use the
          functions you have already coded to write the next ones. The
          difficulty level does not increase by assignment and the
          project has not been structured in any specific way. It is
          similar to a video game, where you can complete quests in
          the order of your choosing and use the loot from the
          previous quests to solve the next ones.

      \item Your project must be written in accordance with the Norm.

      \item Your functions should not quit unexpectedly (segmentation
        fault, bus error, double free, etc) apart from undefined
        behaviors. If this happens, your project will be considered non
        functional and will receive a \texttt{0} during the defence.

        \item All heap allocated memory space must be properly freed
          when necessary.

      \item You must submit a file
        named \texttt{author} containing your username followed by a
        '\textbackslash{}n' at the root of your repository,

      \begin{42console}
$>cat -e author
xlogin$\end{42console}

        \item You must submit a \texttt{C} file for each function you
          create, as well as a \texttt{libft.h} file, which will
          contain all the necessary prototypes as well as
          \texttt{macros} and \texttt{typedefs} you might need. All
          those files must be at the root of your repository.

        \item You must submit a \texttt{Makefile} which will compile
          your source files to a static library \texttt{libft.a}.

        \item Your \texttt{Makefile} must at least contain the rules
          \texttt{\$(NAME)}, \texttt{all}, \texttt{clean},
          \texttt{fclean} et \texttt{re} in the order that you will
          see fit.

        \item Your \texttt{Makefile} must compile your work
          with the flags \texttt{-Wall}, \texttt{-Wextra} and
          \texttt{-Werror}.

        \newpage
        \item Only the following \texttt{libc} functions are allowed :
          \texttt{malloc(3)}, \texttt{free(3)} and \texttt{write(2)},
          and their usage is restricted. See below.

        \item You must include the necessary \texttt{include} system
          files to use one or more of the three authorized functions
          in your \texttt{.c} files. The only additional system
          \texttt{include} file you are allowed to use is
          \texttt{string.h} to have access to the constant
          \texttt{NULL} and to the type \texttt{size\_t}. Everything
          else if forbidden.

        \item We encourage you to create test programs for your
          library even though this work \textbf{won't have to be
            submitted and won't be graded}. It will give you a chance
          to easily test your work and your peers’ work. You will find
          those tests especially useful during your defence. Indeed,
          during defence, you are free to use your tests and/or the
          tests of the peer you are evaluating.

    \end{itemize}



%******************************************************************************%
%                                                                              %
%                                Mandatory part                                %
%                                                                              %
%******************************************************************************%
\chapter{Mandatory part}

    \section{Technical considerations}

        \begin{itemize}

            \item Your \texttt{libft.h} file can contain
              \texttt{macros} and \texttt{typedefs} if needed.

            \item A string must \textbf{ALWAYS} end with a
              \texttt{'\textbackslash{}0'}, even if it is not included
              in the function’s description, unless explicitly stated
              otherwise.

            \item It is forbidden to use global variables.

            \item If you need sub-functions to write a complex
              function, you must define these sub-functions as
              \texttt{static} as stipulated in the Norm.

        \end{itemize}

        \info{Check out this link to find out more about static functions:
          \url{http://codingfreak.blogspot.com/2010/06/static-functions-in-c.html}}

        \begin{itemize}

            \item You must pay attention to your types and wisely use
              the casts when needed, especially when a \texttt{void*}
              type is involved. Generally speaking, avoid implicit
              casts. Example:

            \begin{42ccode}
char    *str;

str = malloc(42 * sizeof(*str));            /* Wrong ! Malloc returns a void * (implicit cast) */
str = (char *) malloc(42 * sizeof(*str));   /* Right ! (explicit cast) */
            \end{42ccode}

        \end{itemize}

    \newpage
    \section{Part 1 - Libc functions}

        In this first part, you must re-code a set of the
        \texttt{libc} functions, as defined in their
        \texttt{man}. Your functions will need to present the same
        prototype and behaviors as the originals. Your functions’
        names must be prefixed by ``\texttt{ft\_}''. For instance
        \texttt{strlen} becomes \texttt{ft\_strlen}.

        \hint {
          Some of the functions’ prototypes you have to re-code
          use the "\texttt{restrict}" qualifier. This keyword is
          part of the \texttt{c99} standard. It is therefore
          forbidden to include it in your prototypes and to compile
          it with the flag \texttt{-std=c99}.
        }

        You must re-code the following functions:

        \vspace{5mm}
        \begin{itemize}\itemsep1mm

            \item \texttt{memset}
            \item \texttt{bzero}
            \item \texttt{memcpy}
            \item \texttt{memccpy}
            \item \texttt{memmove}
            \item \texttt{memchr}
            \item \texttt{memcmp}
            \item \texttt{strlen}
            \item \texttt{strdup}
            \item \texttt{strcpy}
            \item \texttt{strncpy}
            \item \texttt{strcat}
            \item \texttt{strncat}
            \item \texttt{strlcat}
            \item \texttt{strchr}
            \item \texttt{strrchr}
            \item \texttt{strstr}
            \item \texttt{strnstr}
            \item \texttt{strcmp}
            \item \texttt{strncmp}
            \item \texttt{atoi}
            \item \texttt{isalpha}
            \item \texttt{isdigit}
            \item \texttt{isalnum}
            \item \texttt{isascii}
            \item \texttt{isprint}
            \item \texttt{toupper}
            \item \texttt{tolower}

        \end{itemize}

    \newpage
    \section{Part 2 - Additional functions}

        In this second part, you must code a set of functions that are
        either not included in the \texttt{libc}, or included in a
        different form. Some of these functions can be useful to write
        Part 1's functions.\\

    \begin{itemize}\itemsep1pt

            % ft_memalloc
            \item \begin{tabular}{|l|p{11cm}|}
                \hline
                \multicolumn{2}{|c|}{\textbf{ft\_memalloc}}\\
                \hline
                \textbf{Prototype} & \texttt{void
                  *\hspace{5mm}ft\_memalloc(size\_t size);}\\
                \hline
                \textbf{Description} & Allocates (with
                \texttt{malloc(3)}) and returns a ``fresh'' memory area.
                The memory allocated is initialized to \texttt{0}.
                If the allocation fails, the function returns \texttt{NULL}.\\
                \hline
                \textbf{Param. \#1} & The size of the memory that
                needs to be allocated.\\
                \hline
                \textbf{Return value} & The allocated memory area.\\
                \hline
                \textbf{Libc functions} & \texttt{malloc(3)}\\
                \hline
            \end{tabular}

            % ft_memdel
            \item \begin{tabular}{|l|p{11cm}|}
                \hline
                \multicolumn{2}{|c|}{\textbf{ft\_memdel}}\\
                \hline
                \textbf{Prototype} &
                \texttt{void\hspace{5mm}ft\_memdel(void **ap);}\\
                \hline
                \textbf{Description} & Takes as a parameter the address
                of a memory area that needs to be freed with \texttt{free(3)},
                then puts the pointer to \texttt{NULL}.\\
                \hline
                \textbf{Param. \#1} & A pointer's address that needs
                its memory freed and set to \texttt{NULL}.\\
                \hline
                \textbf{Return value} & None.\\
                \hline
                \textbf{Libc functions} & \texttt{free(3)}.\\
                \hline
            \end{tabular}

            % ft_strnew
            \item \begin{tabular}{|l|p{11cm}|}
                \hline
                \multicolumn{2}{|c|}{\textbf{ft\_strnew}}\\
                \hline
                \textbf{Prototype} & \texttt{char
                  *\hspace{5mm}ft\_strnew(size\_t size);}\\
                \hline
                \textbf{Description} & Allocates (with
                \texttt{malloc(3)}) and returns a ``fresh'' string ending
                with \texttt{'\textbackslash{}0'}. Each character of the
                string is initialized at \texttt{'\textbackslash{}0'}.
                If the allocation fails the function returns \texttt{NULL}.\\
                \hline
                \textbf{Param. \#1} & The size of the string to be
                allocated.\\
                \hline
                \textbf{Return value} & The string allocated and initialized
                to \texttt{0}.\\
                \hline
                \textbf{Libc functions} & \texttt{malloc(3)}\\
                \hline
            \end{tabular}

            % ft_strdel
            \item \begin{tabular}{|l|p{11cm}|}
                \hline
                \multicolumn{2}{|c|}{\textbf{ft\_strdel}}\\
                \hline
                \textbf{Prototype} &
                \texttt{void\hspace{5mm}ft\_strdel(char **as);}\\
                \hline
                \textbf{Description} & Takes as a parameter the address
                of a string that need to be freed with \texttt{free(3)},
                then sets its pointer to \texttt{NULL}.\\
                \hline
                \textbf{Param. \#1} & The string's address that needs
                to be freed and its pointer set to \texttt{NULL}.\\
                \hline
                \textbf{Return value} & None.\\
                \hline
                \textbf{Libc functions} & \texttt{Free(3)}.\\
                \hline
            \end{tabular}

            % ft_strclr
            \item \begin{tabular}{|l|p{11cm}|}
                \hline
                \multicolumn{2}{|c|}{\textbf{ft\_strclr}}\\
                \hline
                \textbf{Prototype} &
                \texttt{void\hspace{5mm}ft\_strclr(char *s);}\\
                \hline
                \textbf{Description} & Sets every character of the string to the value
                \texttt{'\textbackslash{}0'}.\\
                \hline
                \textbf{Param. \#1} & The string that needs to be
                cleared.\\
                \hline
                \textbf{Return value} & None.\\
                \hline
                \textbf{Libc functions} & None.\\
                \hline
            \end{tabular}

            % ft_striter
            \item \begin{tabular}{|l|p{11cm}|}
                \hline
                \multicolumn{2}{|c|}{\textbf{ft\_striter}}\\
                \hline
                \textbf{Prototype} &
                \texttt{void\hspace{5mm}ft\_striter(char *s, void
                  (*f)(char *));}\\
                \hline
                \textbf{Description} & Applies the function \texttt{f}
                to each character of the string passed as
                argument. Each character is passed by address to \texttt{f}
                to be modified if necessary.\\
                \hline
                \textbf{Param. \#1} & The string to iterate.\\
                \hline
                \textbf{Param. \#2} & The function to apply to each
                character of \texttt{s}.\\
                \hline
                \textbf{Return value} & None.\\
                \hline
                \textbf{Libc functions} & None.\\
                \hline
            \end{tabular}

            % ft_striteri
            \item \begin{tabular}{|l|p{11cm}|}
                \hline
                \multicolumn{2}{|c|}{\textbf{ft\_striteri}}\\
                \hline
                \textbf{Prototype} &
                \texttt{void\hspace{5mm}ft\_striteri(char *s, void
                  (*f)(unsigned int, char *));}\\
                \hline
                \textbf{Description} & Applies the function \texttt{f}
                to each character of the string passed as
                argument, and passing its index as first argument.
                Each character is passed by address to \texttt{f}
                to be modified if necessary.\\
                \hline
                \textbf{Param. \#1} & The string to iterate.\\
                \hline
                \textbf{Param. \#2} & The function to apply
                to each character of \texttt{s} and its index.\\
                \hline
                \textbf{Return value} & None.\\
                \hline
                \textbf{Libc functions} & None.\\
                \hline
            \end{tabular}

            % ft_strmap
            \item \begin{tabular}{|l|p{11cm}|}
                \hline
                \multicolumn{2}{|c|}{\textbf{ft\_strmap}}\\
                \hline
                \textbf{Prototype} &
                \texttt{char *\hspace{5mm}ft\_strmap(char const *s, char
                  (*f)(char));}\\
                \hline
                \textbf{Description} & Applies the function \texttt{f}
                to each character of the string given as
                argument to create a ``fresh'' new string (with \texttt{malloc(3)})
                resulting from the successive applications of \texttt{f}.\\
                \hline
                \textbf{Param. \#1} & The string to map.\\
                \hline
                \textbf{Param. \#2} & The function to apply
                to each character of \texttt{s}.\\
                \hline
                \textbf{Return value} & The ``fresh'' string created
                from the successive applications of \texttt{f}.\\
                \hline
                \textbf{Libc functions} & \texttt{malloc(3)}\\
                \hline
            \end{tabular}

            % ft_strmapi
            \item \begin{tabular}{|l|p{11cm}|}
                \hline
                \multicolumn{2}{|c|}{\textbf{ft\_strmapi}}\\
                \hline
                \textbf{Prototype} &
                \texttt{char *\hspace{5mm}ft\_strmapi(char const *s, char
                  (*f)(unsigned int, char));}\\
                \hline
                \textbf{Description} & Applies the function \texttt{f}
                to each character of the string passed as
                argument by giving its index as first argument
                to create a ``fresh'' new string (with \texttt{malloc(3)})
                resulting from the successive applications of \texttt{f}.\\
                \hline
                \textbf{Param. \#1} & The string to map.\\
                \hline
                \textbf{Param. \#2} & The function to apply
                to each character of \texttt{s} and its index.\\
                \hline
                \textbf{Return value} & The ``fresh'' string created
                from the successive applications of \texttt{f}.\\
                \hline
                \textbf{Libc functions} & \texttt{malloc(3)}\\
                \hline
            \end{tabular}

            % ft_strequ
            \item \begin{tabular}{|l|p{11cm}|}
                \hline
                \multicolumn{2}{|c|}{\textbf{ft\_strequ}}\\
                \hline
                \textbf{Prototype} &
                \texttt{int\hspace{5mm}ft\_strequ(char const *s1, char
                  const *s2);}\\
                \hline
                \textbf{Description} & Lexicographical comparison between
                \texttt{s1} and \texttt{s2}. If the 2 strings are identical
                the function returns \texttt{1}, or \texttt{0} otherwise.\\
                \hline
                \textbf{Param. \#1} & The first string to be compared.\\
                \hline
                \textbf{Param. \#2} & The second string to be compared.\\
                \hline
                \textbf{Return value} & \texttt{1} or \texttt{0} according to
                if the 2 strings are identical or not.\\
                \hline
                \textbf{Libc functions} & None.\\
                \hline
            \end{tabular}

            % ft_strnequ
            \item \begin{tabular}{|l|p{11cm}|}
                \hline
                \multicolumn{2}{|c|}{\textbf{ft\_strnequ}}\\
                \hline
                \textbf{Prototype} &
                \texttt{int\hspace{5mm}ft\_strnequ(char const *s1, char
                  const *s2, size\_t n);}\\
                \hline
                \textbf{Description} & Lexicographical comparison between
                \texttt{s1} and \texttt{s2} up to \texttt{n} characters
                or until a \texttt{'\textbackslash{}0'} is reached. If the 2 strings
                are identical, the function returns \texttt{1}, or \texttt{0} otherwise.\\
                \hline
                \textbf{Param. \#1} & The first string to be compared.\\
                \hline
                \textbf{Param. \#2} &The second string to be compared.\\
                \hline
                \textbf{Param. \#3} & The maximum number of characters
                to be compared.\\
                \hline
                \textbf{Return value} & \texttt{1} or \texttt{0} according to
                if the 2 strings are identical or not.\\
                \hline
                \textbf{Libc functions} & None.\\
                \hline
            \end{tabular}

            % ft_strsub
            \item \begin{tabular}{|l|p{11cm}|}
                \hline
                \multicolumn{2}{|c|}{\textbf{ft\_strsub}}\\
                \hline
                \textbf{Prototype} & \texttt{char
                  *\hspace{5mm}ft\_strsub(char const *s, unsigned int
                  start, size\_t len);}\\
                \hline
                \textbf{Description} & Allocates (with
                \texttt{malloc(3)}) and returns a ``fresh'' substring
                from the string given as argument. The substring
                begins at index\texttt{start} and is of size \texttt{len}.
                If \texttt{start} and \texttt{len} aren't refering to a
                valid substring, the behavior is undefined. If the
                allocation fails, the function returns \texttt{NULL}.\\
                \hline
                \textbf{Param. \#1} & The string from which create
                the substring.\\
                \hline
                \textbf{Param. \#2} & The start index of the
                substring.\\
                \hline
                \textbf{Param. \#3} & The size of the substring.\\
                \hline
                \textbf{Return value} & The substring.\\
                \hline
                \textbf{Libc functions} & \texttt{malloc(3)}\\
                \hline
            \end{tabular}

            % ft_strjoin
            \item \begin{tabular}{|l|p{11cm}|}
                \hline
                \multicolumn{2}{|c|}{\textbf{ft\_strjoin}}\\
                \hline
                \textbf{Prototype} &
                \texttt{char *\hspace{5mm}ft\_strjoin(char const *s1,
                  char const *s2);}\\
                \hline
                \textbf{Description} & Allocates (with
                \texttt{malloc(3)}) and returns a ``fresh'' string ending
                with \texttt{'\textbackslash{}0'}, result of the concatenation of
                \texttt{s1} and \texttt{s2}. If the allocation fails
                the function returns \texttt{NULL}.\\
                \hline
                \textbf{Param. \#1} & The prefix string.\\
                \hline
                \textbf{Param. \#2} & The suffix string.\\
                \hline
                \textbf{Return value} & The ``fresh'' string result of
                the concatenation of the 2 strings.\\
                \hline
                \textbf{Libc functions} & \texttt{malloc(3)}\\
                \hline
            \end{tabular}

            % ft_strtrim
            \item \begin{tabular}{|l|p{11cm}|}
                \hline
                \multicolumn{2}{|c|}{\textbf{ft\_strtrim}}\\
                \hline
                \textbf{Prototype} &
                \texttt{char *\hspace{5mm}ft\_strtrim(char const *s);}\\
                \hline
                \textbf{Description} & Allocates (with
                \texttt{malloc(3)}) and returns a copy of the string
                given as argument without whitespaces at the beginning or
                at the end of the string. Will be considered as whitespaces
                the following characters \texttt{' '},
                \texttt{'\textbackslash{}n'} and
                \texttt{'\textbackslash{}t'}. If \texttt{s} has no whitespaces
                at the beginning or at the end, the function returns
                a copy of \texttt{s}. If the allocation fails the function returns
                \texttt{NULL}.\\
                \hline
                \textbf{Param. \#1} & The string to be trimed.\\
                \hline
                \textbf{Return value} & The ``fresh'' trimmed string or
                a copy of \texttt{s}.\\
                \hline
                \textbf{Libc functions} & \texttt{malloc(3)}\\
                \hline
            \end{tabular}

            % ft_strsplit
            \item \begin{tabular}{|l|p{11cm}|}
                \hline
                \multicolumn{2}{|c|}{\textbf{ft\_strsplit}}\\
                \hline
                \textbf{Prototype} &
                \texttt{char **\hspace{5mm}ft\_strsplit(char const *s,
                  char c);}\\
                \hline
                \textbf{Description} & Allocates (with
                \texttt{malloc(3)}) and returns an array of ``fresh''
                strings (all ending with \texttt{'\textbackslash{}0'}, including
                the array itself) obtained by spliting \texttt{s} using
                the character \texttt{c} as a delimiter. If the allocation fails the
                function returns \texttt{NULL}. Example :
                \texttt{ft\_strsplit("*hello*fellow***students*", '*')}
                returns the array \texttt{["hello", "fellow",
                    "students"]}.\\
                \hline
                \textbf{Param. \#1} & The string to split.\\
                \hline
                \textbf{Param. \#2} & The delimiter character.\\
                \hline
                \textbf{Return value} & The array of ``fresh'' strings
                result of the split.\\
                \hline
                \textbf{Libc functions} & \texttt{malloc(3)},
                \texttt{free(3)}\\
                \hline
            \end{tabular}

            % ft_itoa
            \item \begin{tabular}{|l|p{11cm}|}
                \hline
                \multicolumn{2}{|c|}{\textbf{ft\_itoa}}\\
                \hline
                \textbf{Prototype} &
                \texttt{char *\hspace{5mm}ft\_itoa(int n);}\\
                \hline
                \textbf{Description} & Allocate (with
                \texttt{malloc(3)}) and returns a ``fresh'' string ending
                with \texttt{'\textbackslash{}0'} representing the
                integer \texttt{n} given as argument. Negative numbers
                must be supported. If the allocation fails, the function
                returns \texttt{NULL}.\\
                \hline
                \textbf{Param. \#1} & The integer to be transformed
                into a string.\\
                \hline
                \textbf{Return value} & The string representing the integer
                passed as argument.\\
                \hline
                \textbf{Libc functions} & \texttt{malloc(3)}\\
                \hline
            \end{tabular}

            % ft_putchar
            \item \begin{tabular}{|l|p{11cm}|}
                \hline
                \multicolumn{2}{|c|}{\textbf{ft\_putchar}}\\
                \hline
                \textbf{Prototype} &
                \texttt{void\hspace{5mm}ft\_putchar(char c);}\\
                \hline
                \textbf{Description} & Outputs the character \texttt{c}
                to the standard output.\\
                \hline
                \textbf{Param. \#1} & The character to output.\\
                \hline
                \textbf{Return value} & None.\\
                \hline
                \textbf{Libc functions} & \texttt{write(2)}.\\
                \hline
            \end{tabular}

            % ft_putstr
            \item \begin{tabular}{|l|p{11cm}|}
                \hline
                \multicolumn{2}{|c|}{\textbf{ft\_putstr}}\\
                \hline
                \textbf{Prototype} &
                \texttt{void\hspace{5mm}ft\_putstr(char const
                  *s);}\\
                \hline
                \textbf{Description} & Outputs the string \texttt{s}
                to the standard output.\\
                \hline
                \textbf{Param. \#1} & The string to output.\\
                \hline
                \textbf{Return value} & None.\\
                \hline
                \textbf{Libc functions} & \texttt{write(2)}.\\
                \hline
            \end{tabular}

            % ft_putendl
            \item \begin{tabular}{|l|p{11cm}|}
                \hline
                \multicolumn{2}{|c|}{\textbf{ft\_putendl}}\\
                \hline
                \textbf{Prototype} &
                \texttt{void\hspace{5mm}ft\_putendl(char const
                  *s);}\\
                \hline
                \textbf{Description} & Outputs the string \texttt{s}
                to the standard output followed by a
                \texttt{'\textbackslash{}n'}.\\
                \hline
                \textbf{Param. \#1} & The string to output.\\
                \hline
                \textbf{Return value} & None.\\
                \hline
                \textbf{Libc functions} & \texttt{write(2)}.\\
                \hline
            \end{tabular}

            % ft_putnbr
            \item \begin{tabular}{|l|p{11cm}|}
                \hline
                \multicolumn{2}{|c|}{\textbf{ft\_putnbr}}\\
                \hline
                \textbf{Prototype} &
                \texttt{void\hspace{5mm}ft\_putnbr(int n);}\\
                \hline
                \textbf{Description} & Outputs the integer \texttt{n}
                to the standard output.\\
                \hline
                \textbf{Param. \#1} & The integer to output.\\
                \hline
                \textbf{Return value} & None.\\
                \hline
                \textbf{Libc functions} & \texttt{write(2)}.\\
                \hline
            \end{tabular}

            % ft_putchar_fd
            \item \begin{tabular}{|l|p{11cm}|}
                \hline
                \multicolumn{2}{|c|}{\textbf{ft\_putchar\_fd}}\\
                \hline
                \textbf{Prototype} &
                \texttt{void\hspace{5mm}ft\_putchar\_fd(char c, int fd);}\\
                \hline
                \textbf{Description} & Outputs the char
                \texttt{c} to the file descriptor \texttt{fd}.\\
                \hline
                \textbf{Param. \#1} & The character to output.\\
                \hline
                \textbf{Param. \#2} & The file descriptor.\\
                \hline
                \textbf{Return value} & None.\\
                \hline
                \textbf{Libc functions} & \texttt{write(2)}.\\
                \hline
            \end{tabular}

            % ft_putstr_fd
            \item \begin{tabular}{|l|p{11cm}|}
                \hline
                \multicolumn{2}{|c|}{\textbf{ft\_putstr\_fd}}\\
                \hline
                \textbf{Prototype} &
                \texttt{void\hspace{5mm}ft\_putstr\_fd(char const
                  *s, int fd);}\\
                \hline
                \textbf{Description} & Outputs the string \texttt{s}
                to the file descriptor \texttt{fd}.\\
                \hline
                \textbf{Param. \#1} & The string to output.\\
                \hline
                \textbf{Param. \#2} & The file descriptor.\\
                \hline
                \textbf{Return value} & None.\\
                \hline
                \textbf{Libc functions} & \texttt{write(2)}.\\
                \hline
            \end{tabular}

            % ft_putendl_fd
            \item \begin{tabular}{|l|p{11cm}|}
                \hline
                \multicolumn{2}{|c|}{\textbf{ft\_putendl\_fd}}\\
                \hline
                \textbf{Prototype} &
                \texttt{void\hspace{5mm}ft\_putendl\_fd(char const
                  *s, int fd);}\\
                \hline
                \textbf{Description} & Outputs the string \texttt{s}
                to the file descriptor \texttt{fd} followed by a
                \texttt{'\textbackslash{}n'}.\\
                \hline
                \textbf{Param. \#1} & The string to output.\\
                \hline
                \textbf{Param. \#2} & The file descriptor.\\
                \hline
                \textbf{Return value} & None.\\
                \hline
                \textbf{Libc functions} & \texttt{write(2)}.\\
                \hline
            \end{tabular}

            % ft_putnbr_fd
            \item \begin{tabular}{|l|p{11cm}|}
                \hline
                \multicolumn{2}{|c|}{\textbf{ft\_putnbr\_fd}}\\
                \hline
                \textbf{Prototype} &
                \texttt{void\hspace{5mm}ft\_putnbr\_fd(int n, int
                  fd);}\\
                \hline
                \textbf{Description} & Outputs the integer \texttt{n}
                to the file descriptor \texttt{fd}.\\
                \hline
                \textbf{Param. \#1} & The integer to print.\\
                \hline
                \textbf{Param. \#2} & The file descriptor.\\
                \hline
                \textbf{Return value} & None.\\
                \hline
                \textbf{Libc functions} & \texttt{write(2)}.\\
                \hline
            \end{tabular}

        \end{itemize}



%******************************************************************************%
%                                                                              %
%                                  Bonus part                                  %
%                                                                              %
%******************************************************************************%
\chapter{Bonus part}

    If you successfully completed the mandatory part, you'll enjoy
    taking it further. You can see this last section as Bonus
    Points.\\

    Having functions to manipulate memory and strings is very
    useful, but you'll soon discover that having functions to
    manipulate lists is even more useful.\\

    You'll use the following structure to represent the links of your
    list. This structure must be added to your \texttt{libft.h}
    file.\\

    \begin{42ccode}
typedef struct      s_list
{
    void            *content;
    size_t          content_size;
    struct s_list   *next;
}                   t_list;
    \end{42ccode}

    Here is a description of the fields of the \texttt{t\_list}
    struct:\\

    \begin{itemize}

        \item \texttt{content} : The data contained in the link. The
          \texttt{void *} allows to store any kind of data.

        \item \texttt{content\_size} : The size of the data stored.
          The \texttt{void *} type doesn't allow you to know the size
          of the pointed data, as a consequence, it is necessary to
          save its size. For instance, the size of the string
          \texttt{"42"} is \texttt{3 bytes} and the 32bits integer
          \texttt{42} has a size of \texttt{4 bytes}.

        \item \texttt{next} : The next link's address or \texttt{NULL}
          if it's the last link.

    \end{itemize}

    \newpage
    The following functions will allow you to manipulate your lists
    more easilly.\\

    \begin{itemize}\itemsep1cm

            % ft_lstnew
            \item \begin{tabular}{|l|p{11cm}|}
                \hline
                \multicolumn{2}{|c|}{\textbf{ft\_lstnew}}\\
                \hline
                \textbf{Prototype} &
                \texttt{t\_list *\hspace{5mm}ft\_lstnew(void const
                  *content, size\_t content\_size);}\\
                \hline
                \textbf{Description} & Allocates (with
                \texttt{malloc(3)}) and returns a ``fresh'' link.
                The variables \texttt{content} and
                \texttt{content\_size} of the new link are
                initialized by \textbf{copy} of the parameters of
                the function. If the parameter \texttt{content} is
                nul, the variable \texttt{content} is initialized to
                \texttt{NULL} and the variable \texttt{content\_size}
                is initialized to \texttt{0} even if the parameter
                \texttt{content\_size} isn't. The variable
                \texttt{next} is initialized to \texttt{NULL}. If
                the allocation fails, the function returns
                \texttt{NULL}.\\
                \hline
                \textbf{Param. \#1} & The content to put in the new
                link.\\
                \hline
                \textbf{Param. \#2} & The size of the content of the
                new link.\\
                \hline
                \textbf{Return value} & The new link.\\
                \hline
                \textbf{Libc functions} & \texttt{malloc(3)},
                \texttt{free(3)}\\
                \hline
            \end{tabular}

            % ft_lstdelone
            \item \begin{tabular}{|l|p{11cm}|}
                \hline
                \multicolumn{2}{|c|}{\textbf{ft\_lstdelone}}\\
                \hline
                \textbf{Prototype} &
                \texttt{void\hspace{5mm}ft\_lstdelone(t\_list **alst,
                  void (*del)(void *, size\_t));}\\
                \hline
                \textbf{Description} & Takes as a parameter a link's
                pointer address and frees the memory of the link's
                content using the function \texttt{del} given as
                a parameter, then frees the link's memory using \texttt{free(3)}.
                The memory of \texttt{next} must not be freed under any circumstance. Finally,
                the pointer to the link that was just freed must be
                set to \texttt{NULL} (quite similar to the function
                \texttt{ft\_memdel} in the mandatory part).\\
                \hline
                \textbf{Param. \#1} & The adress of a pointer to a link
                that needs to be freed.\\
                \hline
                \textbf{Return value} & None.\\
                \hline
                \textbf{Libc functions} & \texttt{free(3)}\\
                \hline
            \end{tabular}

            % ft_lstdel
            \item \begin{tabular}{|l|p{11cm}|}
                \hline
                \multicolumn{2}{|c|}{\textbf{ft\_lstdel}}\\
                \hline
                \textbf{Prototype} &
                \texttt{void\hspace{5mm}ft\_lstdel(t\_list **alst,
                  void (*del)(void *, size\_t));}\\
                \hline
                \textbf{Description} & Takes as a parameter the adress of a pointer to a link
                and frees the memory of this link and
                every successors of that link using the functions \texttt{del}
                and \texttt{free(3)}. Finally the pointer to the
                link that was just freed must be set to \texttt{NULL}
                (quite similar to the function \texttt{ft\_memdel}
                from the mandatory part).\\
                \hline
                \textbf{Param. \#1} & The address of a pointer to the
                first link of a list that needs to be freed.\\
                \hline
                \textbf{Return value} & None.\\
                \hline
                \textbf{Libc functions} & \texttt{free(3)}\\
                \hline
            \end{tabular}

            % ft_lstadd
            \item \begin{tabular}{|l|p{11cm}|}
                \hline
                \multicolumn{2}{|c|}{\textbf{ft\_lstadd}}\\
                \hline
                \textbf{Prototype} &
                \texttt{void\hspace{5mm}ft\_lstadd(t\_list **alst,
                  t\_list *new);}\\
                \hline
                \textbf{Description} & Adds the element \texttt{new}
                at the beginning of the list.\\
                \hline
                \textbf{Param. \#1} & The address of a pointer to the
                first link of a list.\\
                \hline
                \textbf{Param. \#2} & The link to add at the beginning
                of the list.\\
                \hline
                \textbf{Return value} & None.\\
                \hline
                \textbf{Libc functions} & None.\\
                \hline
            \end{tabular}

            % ft_lstiter
            \item \begin{tabular}{|l|p{11cm}|}
                \hline
                \multicolumn{2}{|c|}{\textbf{ft\_lstiter}}\\
                \hline
                \textbf{Prototype} &
                \texttt{void\hspace{5mm}ft\_lstiter(t\_list *lst,
                  void (*f)(t\_list *elem));}\\
                \hline
                \textbf{Description} & Iterates the list \texttt{lst}
                and applies the function \texttt{f} to each link.\\
                \hline
                \textbf{Param. \#1} & A pointer to the
                first link of a list.\\
                \hline
                \textbf{Param. \#2} & The address of a function to
                apply to each link of a list.\\
                \hline
                \textbf{Return value} & None.\\
                \hline
                \textbf{Libc functions} & None.\\
                \hline
            \end{tabular}

            % ft_lstmap
            \item \begin{tabular}{|l|p{11cm}|}
                \hline
                \multicolumn{2}{|c|}{\textbf{ft\_lstmap}}\\
                \hline
                \textbf{Prototype} &
                \texttt{t\_list *\hspace{5mm}ft\_lstmap(t\_list *lst,
                  t\_list * (*f)(t\_list *elem));}\\
                \hline
                \textbf{Description} & Iterates a list \texttt{lst}
                and applies the function \texttt{f} to each link to
                create a ``fresh'' list (using \texttt{malloc(3)})
                resulting from the successive applications of \texttt{f}.
                If the allocation fails, the function returns
                \texttt{NULL}.\\
                \hline
                \textbf{Param. \#1} & A pointer's to the
                first link of a list.\\
                \hline
                \textbf{Param. \#2} & The address of a function to
                apply to each link of a list.\\
                \hline
                \textbf{Return value} & The new list.\\
                \hline
                \textbf{Libc functions} & \texttt{malloc(3)},
                \texttt{free(3)}.\\
                \hline
            \end{tabular}

    \end{itemize}

    \vspace{1cm}

    If you successfully completed both the mandatory and bonus
    sections of this project, we encourage you to add other functions
    that you believe could be useful to expand your library. For
    instance, a version of \texttt{ft\_strsplit} that returns a list
    instead of an array, the function \texttt{ft\_lstfold} similar to
    the function \texttt{reduce} in \texttt{Python} and the function
    \texttt{List.fold\_left} in \texttt{OCaml} (beware of the memory
    leak !). You can add functions to manipulate arrays, stacks,
    files, maps, hashtables, etc. The limit is your imagination.



%******************************************************************************%
%                                                                              %
%                        Submission and peer correction                        %
%                                                                              %
%******************************************************************************%
\chapter{Submission and peer correction}

    Submit your work on your \texttt{GiT} repository as usual.
    Only the work on your repository will be graded.\\

    Once your have completed your defences, Deepthought (the
    ``moulinette'') will grade your work. Your final grade will be
    calculated taking into account your peer-correction grades and
    Deepthought’s grade.\\

    Deepthought will grade your assignments in the order of the
    subject : Part 1, Part 2 and Bonus. One error in one of the
    sections will automatically stop the grading.\\

    Good luck to you and don’t forget your author file !



\end{document}
%******************************************************************************%
